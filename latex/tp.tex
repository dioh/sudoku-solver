\documentclass[a4paper,spanish]{article}

\usepackage{url}
\usepackage{verbatim}
\usepackage{tabu}
\usepackage{float}
\usepackage{cite}
\usepackage[spanish]{cleveref}
\usepackage[colorlinks]{hyperref}

\bibliographystyle{unsrt}

\input{preambulo}

\hyphenation{ob-te-ner}

\begin{document}

\materia{Metaheurísticas}
\submateria{Primer Cuatrimestre de 2013}
\titulo{Trabajo Pr\'{a}ctico Final}
\subtitulo{Un aproximación a la resolución de sudoku}
\abst{s }
\grupo{}
\claves{}
\integrante{Daniel Foguelman}{667/06}{dj.foguleman@gmail.com}


\cfoot{$\thepage$ de \pageref{LastPage}}

\thispagestyle{empty}

\maketitle




%\tableofcontents
\tableofcontents
\pagebreak

%\pagebreak
\section{Resumen}
En el siguiente trabajo práctico resolveremos un problema NP-Completo llamado
Sudoku. Mostraremos una estrategia de resolución exacta y una metaheurística
para resolver el problema. Finalmente mostraremos los resultados de aplicar la
metaheurística.


\section{Introducción}

El juego \emph{Sudoku} es un rompecabezas lógico que trata la ubicación de
números en una grilla de $N^2 x N^2$. El objetivo es completar las celdas
faltantes con valores de $1..N^2$. La grilla se subdivide en $N$ cuadrantes de
$N x N$  cumpliendo con las siguientes reglas:

\begin{enumerate}
    \item Cada fila debe tener los valores de 1 a N una única vez
    \item Cada columna debe tener los valores de 1 a N una única vez
    \item Cada subcuadrante de $NxN$ debe tener los valores de 1 a N una única vez
\end{enumerate}

Un ejemplo de este tipo de problemas tan conocido se observa en la tabla
\ref{tab:sudoku_ejemplo}

\begin{figure}[H]
    \begin{center}
        \begin{tabular}{||c | c | c|| c| c| c|| c| c| c||}
            \hline
            \hline
            &   &   &   &   &   &   &   &   \\
            \hline
            &   &   & 4 &   &   & 3 &   &   \\
            \hline
            6 &   &   &   &   & 7 &   &   &   \\
            \hline
            \hline
            & 5 &   &   &   & 4 &   &   &   \\
            \hline
            4 &   &   &   & 1 &   &   &   & 5 \\
            \hline
            & 3 & 1 & 7 &   & 9 & 8 &   &   \\
            \hline
            \hline
            &   & 3 &   &   &   &   & 5 &   \\
            \hline
            & 6 &   & 8 &   &   & 4 &   & 9 \\
            \hline
            9 & 8 &   &   & 7 &   & 1 & 6 &   \\
            \hline
            \hline

        \end{tabular}
        \label{tab:sudoku_ejemplo}
        \caption{Sudoku ejemplo}
    \end{center}
\end{figure}


La resolución de este problema aplicando reglas lógicas se puede ver en los
metodos Swordfish  o X-Wings.

Notoriamente en los últimos tiempos, han surgido diversos métodos para resolver
de manera heurística el problema.

Los métodos de resolución exacta para este problema con \emph{Fuerza bruta}
consiste en asignar posibles valores iterativamente a las celdas e ir
verificando si el sudoku verifica las reglas a medida que se siguen completando
los blancos.

Este esquema es costoso y a menudo se utiliza con backtracking.


En la actualidad se han encontrado avances en la resolución de Sudoku utilizando
algorítmos genéticos \cite{mantere2007solving}, Búsqueda harmónica
\cite{geem2007harmony}, SAT-Solving \cite{lynce2006sudoku} y finalmente
Simulated Annealing \cite{lewis2007metaheuristics}.


En este trabajo nos centraremos en un acercamiento de búsqueda local comparando
los métodos Búsqueda local\cite{aarts2003local} y Threshold
Accepting\cite{dueck1990threshold} con lo expresado por el autor de
\cite{lewis2007metaheuristics}.

\section{Búsqueda Local}

Supongamos que tenemos un problema al que queremos encontrarle una solución
óptima, este problema toma diversos parámetros y ante cada instanciación
devuelve una solución que puede ser correcta en terminos de las restricciones
del problema, puede ser óptima en terminos de ser la mejor solucion, o puede ser
ni una ni la otra. En este último caso, deberemos evaluar si siguiendo por esa
solución parcial del problema alcanzaremos una solución que sea a la vez
correcta y óptima.

Las soluciones factibles definen el espacio de soluciones que verifican ciertas
condiciones, dentro de las que nos mantendremos si violar las restricciones del
problema.

Una solución óptima puede no ser siempre alcanzable y en ese caso se busca la
mejor solución. Se dice que una solución $S_1$ es mejor que otra solución $S_2$ si
dada una función objetivo  que toma instancias de solución nos permite
compararlas para determinar si una es mejor que la otra.
\begin{equation}
    f(S_1) < f(S_2) 
\end{equation}

Para poder recorrer el espacio de soluciones del problema, definimos una función
de vecindad que dada una solución inicial $S_i$ del problema nos devuelve una
solución $S$.


El método de Búsqueda local entonces, itera sobre el espacio de soluciones
buscando en cada paso obtener una mejor solución parcial hasta alcanzar un
mínimo. Luego de la estabilización, este se detiene.

¿Qué sucede si el método no alcanzó el mínimo absoluto? Este se quedó estancado
en una solución que no se encuentra cercana al valor óptimo y se estancará.

Es por este motivo que se determinan optimizaciones a Búsqueda Local, como
pueden ser Simmulating annealing, Hill Climbing, Tabú Search o Threshold
Accepting.

\emph{Threshold Accepting} en contraposición a \emph{Búsqueda Local} permite
movimientos a soluciones cuya función objetivo no necesariamente mejora. Esto le
permite salir de los mínimos locales. Al igual que en búsqueda local, es
necesario definir la función objetivo y la función para obtener el próximo
vecino, pero también requiere definir los umbrales ante los cuales se aceptan
evaluaciones de la función de objetivo que son peores a la solución actual.
A medida que pasan las iteraciones, el valor umbral disminuye haciendo que el
algoritmo se comporte identicamente a búsqueda local.


\section{Desarrollo}

El problema 

\bibliography{citas}

\end{document}
