\documentclass[a4paper,spanish]{article}

\usepackage{url}
\usepackage{verbatim}
\usepackage{tabu}
\usepackage{float}
\usepackage{cite}
\usepackage[spanish]{cleveref}
\usepackage[colorlinks]{hyperref}

\bibliographystyle{unsrt}

\usepackage[activeacute]{babel}
\usepackage{fancyhdr}
\usepackage{ecarat}
\usepackage{graphicx}
%\usepackage{amssymb}
\usepackage{amssymb,amsmath}
%\usepackage[all]{xy}
\usepackage{graphicx}
\usepackage{listings}
%\usepackage{pseudocode}
\usepackage[utf8]{inputenc}
%\usepackage{wrapfig}
\usepackage{lastpage}
\usepackage{multicol}
\usepackage{url}
\usepackage{color}
\usepackage{array}
\usepackage{url}
\usepackage{listings}
\usepackage{sverb}



\graphicspath{
{./images/}
}

\usepackage{xifthen}% provides \isempty test
%~ \usepackage{ulem}%para subrayados
\usepackage{dashrule}%para subrayados

%~ \usepackage{relacion}

\oddsidemargin 0in
\textwidth 6.5in
\topmargin 0in
\addtolength{\topmargin}{-.4in}
\addtolength{\leftmargin}{-.5in}
\textheight 10in
\parskip=1ex
\pagestyle{fancy}
\newcommand{\real}{\hbox{\bf R}}
\newcommand{\prima}{^{\prime}}
\newcommand{\tab}{\hspace*{1cm}}
%Cosas que se usan en el enunciado
% \parindent = 0 pt
% \parskip = 11 pt

\usepackage{a4wide}
\newcommand{\erf}{\hbox{erf}}
\newcommand\abs[1]{\ensuremath{\left| {#1} \right|}}
%~ \addtolength{\textheight}{1cm}

\lstdefinelanguage{Smalltalk}{ 
	morekeywords={true,false,self,super,nil}, 
	sensitive=true, 
	morecomment=[s]{"}{"}, 
	morestring=[d]', 
	style=SmalltalkStyle 
} 
\lstdefinestyle{SmalltalkStyle}{ 
	literate={:=}{{$\gets$}}1{^}{{$\uparrow$}}1 
} 
\lstset {%
     language=Smalltalk,
     frame=l,
     framerule=1pt,
     aboveskip=1em,
     framextopmargin=3pt,
     framexbottommargin=3pt,
     framexleftmargin=0.4cm,
     framesep=0pt,
     rulesep=.4pt,
     %~ backgroundcolor=\color{gris95},
     rulesepcolor=\color{red},
     tabsize=4,
     %
     stringstyle=\sffamily,
     showstringspaces = false,
     basicstyle=\footnotesize\sffamily,
     identifierstyle=,
     commentstyle=\em, %\sffamily\color{gris50},
     keywordstyle=\bfseries,
     commentstyle=\scriptsize\sffamily,
     breaklines=true,
     breakatwhitespace=true,
     morekeywords={setof},
   }

\usepackage{longtable}


\hyphenation{ob-te-ner}

\begin{document}

\materia{Metaheurísticas}
\submateria{Primer Cuatrimestre de 2013}
\titulo{Trabajo Pr\'{a}ctico Final}
\subtitulo{Un aproximación a la resolución de sudoku}
\abst{s }
\grupo{}
\claves{}
\integrante{Daniel Foguelman}{667/06}{dj.foguleman@gmail.com}


\cfoot{$\thepage$ de \pageref{LastPage}}

\thispagestyle{empty}

\maketitle




%\tableofcontents
\tableofcontents
\pagebreak

%\pagebreak
\section{Resumen}
En el siguiente trabajo práctico resolveremos un problema NP-Completo llamado
Sudoku. Mostraremos una estrategia de resolución exacta y una metaheurística
para resolver el problema. Finalmente mostraremos los resultados de aplicar la
metaheurística.


\section{Introducción}
\label{sec:intro}

El juego \emph{Sudoku} es un rompecabezas lógico que trata la ubicación de
números en una grilla de $N^2 x N^2$. El objetivo es completar las celdas
faltantes con valores de $1..N^2$. La grilla se subdivide en $N$ cuadrantes de
$N x N$  cumpliendo con las siguientes reglas:

\begin{enumerate}
        \label{enum:principios}

    \item Cada fila debe tener los valores de 1 a N una única vez
    \item Cada columna debe tener los valores de 1 a N una única vez
    \item Cada subcuadrante de $NxN$ debe tener los valores de 1 a N una única vez
\end{enumerate}

Un ejemplo de este tipo de problemas tan conocido se observa en la tabla
\ref{tab:sudoku_ejemplo}

\begin{figure}[H]
    \begin{center}
        \begin{tabular}{||c | c | c|| c| c| c|| c| c| c||}
            \hline
            \hline
            &   &   &   &   &   &   &   &   \\
            \hline
            &   &   & 4 &   &   & 3 &   &   \\
            \hline
            6 &   &   &   &   & 7 &   &   &   \\
            \hline
            \hline
            & 5 &   &   &   & 4 &   &   &   \\
            \hline
            4 &   &   &   & 1 &   &   &   & 5 \\
            \hline
            & 3 & 1 & 7 &   & 9 & 8 &   &   \\
            \hline
            \hline
            &   & 3 &   &   &   &   & 5 &   \\
            \hline
            & 6 &   & 8 &   &   & 4 &   & 9 \\
            \hline
            9 & 8 &   &   & 7 &   & 1 & 6 &   \\
            \hline
            \hline

        \end{tabular}
        \label{tab:sudoku_ejemplo}
        \caption{Sudoku ejemplo}
    \end{center}
\end{figure}


La resolución de este problema aplicando reglas lógicas se puede ver en los
metodos Swordfish  o X-Wings.

Notoriamente en los últimos tiempos, han surgido diversos métodos para resolver
de manera heurística el problema.

Los métodos de resolución exacta para este problema con \emph{Fuerza bruta}
consiste en asignar posibles valores iterativamente a las celdas e ir
verificando si el sudoku verifica las reglas a medida que se siguen completando
los blancos.

Este esquema es costoso y a menudo se utiliza con backtracking.


En la actualidad se han encontrado avances en la resolución de Sudoku utilizando
algorítmos genéticos \cite{mantere2007solving}, Búsqueda harmónica
\cite{geem2007harmony}, SAT-Solving \cite{lynce2006sudoku} y finalmente
Simulated Annealing \cite{lewis2007metaheuristics}.


En este trabajo nos centraremos en un acercamiento de búsqueda local comparando
los métodos Búsqueda local\cite{aarts2003local} y Threshold
Accepting\cite{dueck1990threshold} con lo expresado por el autor de
\cite{lewis2007metaheuristics}.

\section{Búsqueda Local}

Supongamos que tenemos un problema al que queremos encontrarle una solución
óptima, este problema toma diversos parámetros y ante cada instanciación
devuelve una solución que puede ser correcta en terminos de las restricciones
del problema, puede ser óptima en terminos de ser la mejor solucion, o puede ser
ni una ni la otra. En este último caso, deberemos evaluar si siguiendo por esa
solución parcial del problema alcanzaremos una solución que sea a la vez
correcta y óptima.

Las soluciones factibles definen el espacio de soluciones que verifican ciertas
condiciones, dentro de las que nos mantendremos si violar las restricciones del
problema.

Una solución óptima puede no ser siempre alcanzable y en ese caso se busca la
mejor solución. Se dice que una solución $S_1$ es mejor que otra solución $S_2$ si
dada una función objetivo  que toma instancias de solución nos permite
compararlas para determinar si una es mejor que la otra.
\begin{equation}
    f(S_1) < f(S_2) 
\end{equation}

Para poder recorrer el espacio de soluciones del problema, definimos una función
de vecindad que dada una solución inicial $S_i$ del problema nos devuelve una
solución $S$.


El método de Búsqueda local entonces, itera sobre el espacio de soluciones
buscando en cada paso obtener una mejor solución parcial hasta alcanzar un
mínimo. Luego de la estabilización, este se detiene.

¿Qué sucede si el método no alcanzó el mínimo absoluto? Este se quedó estancado
en una solución que no se encuentra cercana al valor óptimo y se estancará.

Es por este motivo que se determinan optimizaciones a Búsqueda Local, como
pueden ser Simmulating annealing, Hill Climbing, Tabú Search o Threshold
Accepting.

\emph{Threshold Accepting} en contraposición a \emph{Búsqueda Local} permite
movimientos a soluciones cuya función objetivo no necesariamente mejora. Esto le
permite salir de los mínimos locales. Al igual que en búsqueda local, es
necesario definir la función objetivo y la función para obtener el próximo
vecino, pero también requiere definir los umbrales ante los cuales se aceptan
evaluaciones de la función de objetivo que son peores a la solución actual.
A medida que pasan las iteraciones, el valor umbral disminuye haciendo que el
algoritmo se comporte identicamente a búsqueda local.


\section{Desarrollo}


En la sección anterior discutimos dos metaheurísticas de búsqueda local, ambas
precisan la definición de la función objetivo y la función de vecindad.

\subsection{Solución inicial}

Nuestra solución inicial está definida de manera que no se rompa la regla 3
definida en las reglas de sudoku. De esta manera asignaremos valores del 1 al
$N^2$ en cada cuadrante de $NxN$ sin que se repitan.

Así buscaremos mejorar nuestra solución hasta que se cumplan las reglas 1 y 2. 

\subsection{Función objetivo}

Definiremos la función objetivo como la suma de la cantidad de elementos
repetidos por fila y columna.  
Esta función vale cero cuando la solución del sudoku es correcta.

\subsection{Función de vecindad}

Definimos nuestra función de vecindad que toma una matriz sudoku como fue
definida en \ref{sec:intro}.
Intercambiaremos un valor (seleccionado al azar) dentro de la matriz por otro
dentro del mismo cuadrante, también seleccionado al azar. Tendremos cuidado en
no intercambiar los valores que ya vinieron fijos en el sudoku de entrada.


\subsection{Casos de prueba}


Para evaluar el comportamiento de ambos algorítmos utilizamos casos de prueba
generados con la aplicación \emph{qqwing}. Este nos permite generar tableros
sudoku de distinta dificultad.

Para esto generamos veinte tableros de cada tipo de dificultad para poder
evaluar el comportamiento de ámbos algorítmos y la asignación reportada por cada
uno.

Para las figuras  \ref{img:histo_easy}, \ref{img:histo_med} y
\ref{img:histo_hard}, corrimos ambos algoritmos con un set de 20 instancias de
sudoku generadas al azar para determinar cual es la eficacia del altorítmo para
3000 iteraciones (este valor se determinó empiricamente en base a esperimentos
anteriores).
Podemos observar la cantidad de soluciones óptimas que ha obtenido cada
algorítmo. Observamos que Threshold Accepting supera a Búsqueda local por más de
un 50\%.

También se observa que la convergencia de las soluciones es buena, es decir que
todos los valores que se han obtenido están en un entorno cercano a la solución
óptima.


\begin{center}
    \begin{figure}[H]
        \includegraphics[width=\textwidth]{imgs/problemas_easy_histo.png}
        \caption{Histograma asignación Búsqueda local (LS) y Threshold
        Accepting (TA) para el conjunto de problemas fácil)}
        \label{img:histo_easy}
    \end{figure}
\end{center}
\begin{center}
    \begin{figure}[H]
        \includegraphics[width=\textwidth]{imgs/problemas_med_histo.png}
        \caption{Histograma asignación Búsqueda local (LS) y Threshold
        Accepting (TA) para el conjunto de problemas mediano}
        \label{img:histo_med}
    \end{figure}
\end{center}
\begin{center}
    \begin{figure}[H]
        \includegraphics[width=\textwidth]{imgs/problemas_hard_histo.png}
        \caption{Histograma asignación Búsqueda local (LS) y Threshold
        Accepting (TA) para el conjunto de problemas difícil}
        \label{img:histo_hard}
    \end{figure}
\end{center}

También observamos que en \ref{img:histo_med} la obtención del óptimo es mucho
mayor. Cosa que no se reproduce en \ref{img:histo_easy} ni en
\ref{img:histo_hard}. Esto nos muestra que no depende de la facilidad de
resolución del problema (en términos lógico deductivos) sino más bien en la
distribución de los vecinos.

Si observamos el comportamiento de los algorítmos con un tablero fijo y variando
la cantidad de iteraciones que precisa para alcanzar el óptimo obtenemos las
figuras \ref{img:prog_ls} y \ref{img:prog_ta}. Para esto se generaron 50
corridas de asignacion variando la cantidad de iteraciones admitidas para el
algorítmo de 100 a 3000.

Las figuras presentan las medias y varianzas observacionales de los
experimentos.


\begin{center}
    \begin{figure}[H]
        \includegraphics[width=\textwidth]{imgs/BLsol_progresion.png}
        \caption{Progresión de parámetro máxima cantidad de iteraciones para Búsqueda Local}
        \label{img:prog_ls}
    \end{figure}
\end{center}


\begin{center}
    \begin{figure}[H]
        \includegraphics[width=\textwidth]{imgs/TAsol_progresion.png}
        \caption{Progresión de parámetro máxima cantidad de iteraciones para
        Threshold Accepting}
        \label{img:prog_ta}
    \end{figure}
\end{center}

Determinamos que los algorítmos se estabilizan luego de las 3000 iteraciones
aunque la varianza para la asignación es muy elevada. Esto nos indica que el
espacio de soluciones es muy grande y por lo tanto en algunos casos los mínimos locales
alcanzados son dificiles de sobrellevar.



\section{Discusión}

En este trabajo experimentamos con la resolución del problema de Sudoku con dos
heurísticas distintas de búsqueda local. Utilizamos el esquema \textit{puro} de
búsqueda local y una modificación \emph{Threshold Accepting} para contrastar los
resultados y medir la eficacia de cada algorítmo.

Encontramos que Threshold Accepting resuelve más instancias del problema que
Búsqueda local con una penalidad en cantidad de operaciones.

Encontramos que no hay relación entre la dificultad de resolver los problemas de
manera lógico deductivo con la resolución algorítmica utilizando estos métodos.

La facilidad de implementación de estos algorítmos los hacen excelentes
candidatos para poder determinar una solución inicial para un problema
combinatorio de estas características. Sin embargo es probable que no encuentren
el mínimo global dado las limitaciones que presentan en su diseño.


Para esta instancia particular, búsqueda local mejora la solución del problema
de manera $e^-1$.
Para Threshold Accepting 

\bibliography{citas}

\end{document}
